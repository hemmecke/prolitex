\documentclass[a4paper]{article}
\usepackage{prolitex}
\usepackage[x11names]{xcolor}
\lstset{backgroundcolor=\color{SlateBlue3!10!gray!10!white},
  captionstyle=\bfseries\color{SlateBlue4},
  language=Awk}
\usepackage{hyperref}

\newcommand{\packagename}[1]{{\normalfont\sffamily#1}}
\newcommand\ProLiteX{\textsf{ProLiteX}}
\newcommand\prolitex{\packagename{prolitex}}
\newcommand{\ie}{i.\,e.}
\lstnewenvironment{myverbatim}%
  {\lstset{basicstyle=\scriptsize\ttfamily,gobble=2}}%
  {}
% We must use basicstyle=\scriptsize in order to get about 80
% charactes per line in the code chunk.
\lstset{basicstyle=\scriptsize}

% Similar to the definition of the description environment in
% article.cls.
\newenvironment{ttdescription}
  {\list{}{\labelwidth0pt \itemindent-\leftmargin
      \def\makelabel##1{\hspace\labelsep
        \normalfont\ttfamily ##1:}}}
  {\endlist}
%%%%%%%%%%%%%%%%%%%%%%%%%%%%%%%%%%%%%%%%%%%%%%%%%%%%%%%%%%%%%%%%%%%
\begin{document}
\title{\ProLiteX}
\date{17-May-2010}
\author{Ralf Hemmecke}
\maketitle
\begin{abstract}
  The \ProLiteX{} project consists of a collection of tools to
  supports literate programming. In contrast to most other tools
  \ProLiteX{} has no weave command to extract a \texttt{.tex} document
  from the literate sources. Instead, \ProLiteX{} defines an ordinary
  \LaTeX{} \texttt{chunk} environment into which program code is to be
  written. Due to that, literate sources are pure \LaTeX{} files.

  As the tangle part, \ie, extracting the source code from the
  literate sources, \ProLiteX{} provides an \packagename{awk} script.
\end{abstract}

\section{Introduction}

\section{Implementation}

Since this file is a literate program (\ie, it contains documentation
and source code) of \ProLiteX's tangle command, there is a little
bootstrapping problem. \ProLiteX's tangle program cannot yet be
assumed to exist. We, therefore, do not use code junk references here
and write the program in a linear fashion so that just extracting the
lines inbetween \verb|\begin{chunk}| and \verb|\end{chunk}| will give
the respective source code.

This extraction can easily be done by piping this file through the
following simple \packagename{awk} command.
\begin{myverbatim}
  awk '/^\\begin{chunk}/{c=1;next};/^\\end{chunk}/{c=0;next};c'
\end{myverbatim}

Since this \packagename{awk} file should be a script, we start with an
appropriate first line. Note that we assume that the executable
\texttt{awk} lives under \texttt{/usr/bin}. That might not everywhere
be the case. Maybe our \texttt{configure} script should deal with
this issue.
\begin{chunk}{*}
#!/usr/bin/awk -f
\end{chunk}

Let's have an error function that aborts the program.
\begin{chunk}{*}
function error(msg) {
  print "==========================================="
  print "Error in line " NR " of " FILENAME "."
  print msg
  exit
}
\end{chunk}



\subsection{Overview}

The implementation is done for two cases (a) where the chunkref
delimiters do not change and (b) where they can change due to a
\verb'\lstset' command or optional parameters in square brackets
immediately after \verb%\begin{cunk}%.

The flow of the \packagename{awk} program is simple and done in three
phases.

\begin{enumerate}
\item First, we collect all chunk data in an array \texttt{data}. Each
  chunk environment from the source file becomes a separate entry. No
  concatenation of chunks is done. And no chunk references are
  resolved.

\item The first phase ends when the \texttt{END} clause of the
  \packagename{awk} program will be executed, \ie, when the whole
  source file has been read.
%
  The second phase expands the chunks and determines which chunks are
  root chunks.

\item Finally, in the third phase, the desired data is written to stdout
  or to the respective files.
\end{enumerate}
\subsection{Phase 1}
\subsubsection{Globabl Variables in Phase 1}

We basically store all the information about the chunks in an array
\texttt{data}. Additionally, each unique chunkname is collected as an
index in the array \texttt{chunk}.

We list here the all the global variables.
\begin{ttdescription}
\item[bdelimglobal \textrm{and} edelimglobal] These variables are used
  to store the global settings for the chunkref delimiters.
  % Because in the following chunk we write <<, we must redefine
  % the default chunkref delimiter. Since this file does actually use
  % chunkref delimiters, it does not matter to what delimiters we
  % redefine the default. It just should not appear inside any of the
  % following chunk environments.
  \lstset{chunkref={@<@}{@>@}}
\begin{chunk}{*}
{
  bdelimglobal = "<<"
  edelimglobal = ">>"
}
\end{chunk}
If no other delimiters are given, these will be used. 
By a command of the form
\begin{myverbatim}
  \lstset{chunkref={<[}{]>}}
\end{myverbatim}
the global chunkref delimiters can be changed for the following chunk
enviroments to \texttt{<[} and \texttt{]>}.


\item[data] Input lines between
\begin{myverbatim}
  \begin{chunk}{chunkname}
\end{myverbatim}
and 
\begin{myverbatim}
  \end{chunk}
\end{myverbatim}
are collected into the array \texttt{data}.%
\footnote{In general, there can be an optional argument enclosed in
  square brackets between \texttt{\{chunk\}} and
  \texttt{\{chunkname\}}.}
%
Note that both of the environment delimiters are supposed to start at
the beginning of the line.

A detailed description follows below.

\item[inchunk] This variable is true if we are inside a chunk, \ie, if
  we must collect the lines.

\item[chunkname] The name of the current chunk.

\item[part] The part number of the current chunk. The smallest number
  is 1.

\item[lines] Not really necessary, but is used to collect all chunk
  lines before they are stored into
  \texttt{data[chunkname,part,"lines"]}.

\item[chunkref] Not really necessary, but is used to remember whether
  there is a chunk reference inside the current chunk. At the end of
  the current chunk part, its value is stored into
  \texttt{data[chunkname,part,"chunkref?"]}.
%
  A non-zero value indicates that this chunk contains an unexpanded
  chunk reference.
\end{ttdescription}


\subsubsection{The \texttt{data} Array}
Let us describe the contents of the array \texttt{data} in more
detail.

Since there might be different chunks with the same name and we plan
for changes of the chunkref delimiters, we define for each
\texttt{chunkname} the following array indices, where
\texttt{chunkname} and \texttt{part} have the meaning from above.

\begin{ttdescription}
\item[chunkname,"parts"] The number of chunks with the same name.

\item[chunkname,part,"lineno"] The source line number of the first
  chunk line. This will be used to generate \texttt{\#line}
  directives.

\item[chunkname,part,"lines"] The content of a part of a chunk. All
  lines are concatenated and stored in this entry. The trailing
  newline character of the last line is removed.

\item[chunkname,part,"bdelim"] The starting chunkref delimiter
  sequence that is valid for this chunk.

\item[chunkname,part,"edelim"] The ending chunkref delimiter sequence
  that is valid for this chunk.

\item[chunkname,part,"chunkref?"] This is true (\ie, greater than 0)
  if in this part of the chunk there are references to other chunks.
\end{ttdescription}

\subsubsection{Code for Phase 1}

We start with a first simple implementation where we do not consider
any options that change the default setting for the chunkref
delimiters, which are \texttt{<<} and \texttt{>>}.

Let's first treat the simple case to start a chunk, \ie, the chunk has
no optional parameter. We don't allow any character appearing after
the closing brace that ends the chunkname.
\begin{chunk}{*}
/^\\begin{chunk}{.+}$/ {#$
  inchunk = 1
  chunkname = substr($0,15,length($0)-15)
  part = ++data[chunkname,"parts"]
  if (part == 1) {
    chunk[chunkname] = 0 # assume that this is a root chunk
  }
  data[chunkname,part,"bdelim"] = bdelimglobal
  data[chunkname,part,"edelim"] = edelimglobal
  data[chunkname,part,"lineno"] = NR+1
  lines = ""
  chunkref = 0 #no chunkref found so far
  next
}
\end{chunk}

Eventually, we switch off the chunk treatment. It only makes sense if
we are inside a chunk. Otherwise we abort.
\begin{chunk}{*}
/^\\end{chunk}/ {
  if (inchunk == 0) {
    error("Have found \\end{chunk} without \\begin{chunk}.")
  }
  data[chunkname,part,"lines"] = substr(lines,2) # remove initial \n
  data[chunkname,part,"chunkref?"] = chunkref
  data[chunkname,"chunkref?"] += chunkref
  inchunk=0
  next
}
\end{chunk}

It is also easy to actually treat the chunk, \ie, when we are inside a
chunk and read the chunk lines. Note that this chunk must come after
the part that treats the end of a code chunk.
\begin{chunk}{*}
inchunk {
  chunkref += index($0, data[chunkname,part,"bdelim"]) #$
  lines = lines "\n" $0 #$
  next
}
\end{chunk}


\subsection{Phase 2}
\subsubsection{Globabl Variables in Phase 2}

In addition to the indices of the \texttt{data} array given above, the
following indices are used.
%
Phase 2 will only read the values from the indices in Phase 1, but
never change them.
\begin{ttdescription}
\item[chunkname,"lines"] The whole chunk, \ie, all parts concatenated
  and properly expanded. This is only defined if there is an already
  computed expanded chunk.
\end{ttdescription}

Since \packagename{awk} does not allow to return objects, we use
global variables. The following global variables are set by the
function \texttt{haschunkref} as a side effect and are not modified by
any other function.
\begin{ttdescription}
\item[before] to the string part before matching \texttt{bdelim},
\item[chunkrefname] to the name that has been found,
\item[after] to the string part after the closing \texttt{edelim},
\item[afterindex] to the index where the string \texttt{after} begins
  in the originally given string.
\end{ttdescription}


\subsubsection{Code for Phase 2}

Let us define some auxiliary functions.

The following function \texttt{haschunkref} returns the index where a
chunk reference has been found. It is the first index of the string
\texttt{bdelim}. If no chunk reference can be found this function
returns 0.

As a side effect, this variable sets the global variables
\texttt{before}, \texttt{chunkrefname}, \texttt{after},
\texttt{afterindex}. See their description above.

\begin{chunk}{*}
function haschunkref(s, bdelim, edelim,
  # local variables
  b, c, e) {
  #
  b = index(s, bdelim)
  if (b == 0) { return 0 }
  before = substr(s, 1, b-1)
  c = b + length(bdelim)
  e = index(substr(s, c), edelim)
  if (e == 0) {
    error("Closing chunkref delimiter not found.")
  }
  chunkrefname = substr(s, c, e-1)
  afterindex = c+e+length(edelim)-1
  after = substr(s, afterindex)
  return b
}
\end{chunk}

The function \texttt{indentchunk} indents every line in the parameter
\texttt{lines} by $n$ spaces. We simply replace each newline by a
newline followed by $n$ spaces.
\begin{chunk}{*}
function indentchunk(n, lines,
  # local variables
  f, indent) {
  #
  f = "%" n "s"
  indent = sprintf("%" n "s", "")
  gsub(/\n/, "\n" indent, lines)
  return lines
}
\end{chunk}

Including chunk in one line \texttt{aline} means to determine the
index and name of the chunk reference and replace it by the
corresponding (properly indented) code chunk. The chunk delimiters
have to be given when the function is invoked.
\begin{chunk}{*}
function includechunks(aline, bdelim, edelim,
  # local variables
  i, n, s) {
  #
  while (i = haschunkref(aline, bdelim, edelim)) {
    s = s before indentchunk(n+i-1,data[chunkrefname,"lines"])
    n += afterindex-1
    aline = after
  }
  return s aline
}
\end{chunk}

When we enter Phase 2, no chunk has yet been expanded. Since one chunk
may comprise several parts that have to be expanded separately and
then concatenated, we have two functions that are mutually call each
other. While \texttt{expandchunkrefsN} recursively expands the chunk
references of a chunk part with name \texttt{chunkname} and number
\texttt{part}, the function \texttt{expandchunkrefs} expands all chunk
parts, concatenates them and stores these lines into the variable
\texttt{data[chunkname,"lines"]}.

\begin{quote}
  \textbf{Warning:} There is no code that detects recursion in the
    chunks. The chunk structure is supposed to form a tree (or rather
    a forest, since there might be several roots).
\end{quote}

Due to passing arrays by reference in \packagename{awk}, we basically
cannot use arrays in recursive invocations of functions. Therefore, we
parse the given chunk lines twice. In the first round, we do not yet
compute the resulting lines, but rather only recursively expand the
chunks that are referenced from the current chunk. In the second round
we can simply include the (already) expanded lines from the referenced
chunks.
\begin{chunk}{*}
function expandchunkrefsN(chunkname,part,
  # local variables
  chunkdata, bdelim, edelim, n, s, i, line) {
  #
  chunkdata = data[chunkname, part, "lines"]
  # return early if the chunk does not reference other chunks
  if (data[chunkname, part, "chunkref?"] == 0) { return chunkdata }
  # First round.
  bdelim = data[chunkname, part, "bdelim"]
  edelim = data[chunkname, part, "edelim"]
  while (haschunkref(chunkdata, bdelim, edelim)) {
    chunkdata = after
    expandchunkrefs(chunkrefname)
  }
  # Second round. Now every subchunk is properly expanded. We include them.
  n = split(data[chunkname,part,"lines"],line,"\n")
  if (n == 0) {return ""} # empty chunk
  s = includechunks(line[1], bdelim, edelim)
  for(i=2; i<=n; i++) {s = s "\n" includechunks(line[i], bdelim, edelim)}
  return s
}
\end{chunk}

And now we expand the various chunk parts and store these lines
concatenated into \texttt{data[chunkname,"lines"]}.
\begin{chunk}{*}
function expandchunkrefs(chunkname,
  # local variables
  parts, s, k) {
  # return early if the chunk has already been expanded
  if ((chunkname,"lines") in chunk) { return data[chunkname,"lines"] }
  parts = data[chunkname,"parts"]
  s = expandchunkrefsN(chunkname,1)
  for (k=2; k<=parts; k++) {s = s "\n" expandchunkrefsN(chunkname,k)}
  data[chunkname,"lines"] = s
  return s
}
\end{chunk}

The following piece of code just triggers the expansion and output of
all the chunks.
\begin{chunk}{*}
END {
  for (i in chunk) { print expandchunkrefs(i) }
}
\end{chunk}
\end{document}
