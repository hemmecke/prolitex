% \iffalse
%%
%% prolitex.dtx
%% Copyright (C) 2010,  Ralf Hemmecke <ralf@hemmecke.de>
%%
% This work may be distributed and/or modified under the
% conditions of the LaTeX Project Public License, either version 1.3
% of this license or (at your option) any later version.
% The latest version of this license is in
%   http://www.latex-project.org/lppl.txt
% and version 1.3 or later is part of all distributions of LaTeX
% version 2005/12/01 or later.
%
% This work has the LPPL maintenance status `maintained'.
%
% The Current Maintainer is Ralf Hemmecke <ralf@hemmecke.de>.
%
% This work consists of the files prolitex.dtx and prolitex.ins
% and the derived file prolitex.sty.
%%
%<*driver>
% The following construction with defining |\ProLiteXDocumentation|
% and calling it only if |\documentclass| is not equal to
% |\@twoclasseserror|, i.e., if |\documentclass| has not yet been
% executed, serves the sole purpose of making this .dtx file a proper
% .sty file.
% \url{http://groups.google.com/group/comp.text.tex/browse_thread/thread/507f50d03a2f8ed2?hl=en#}
\def\ProliteXDocumentation{%
  \documentclass[a4paper]{ltxdoc}
  \DisableCrossrefs
  \usepackage{url}
  \begin{document}
  \DocInput{prolitex.dtx}
  \end{document}}
\ifx\documentclass\@twoclasseserror\else\expandafter\ProliteXDocumentation\fi
%</driver>
% \fi
%
%^^A Some definitions for the documentation of ProLiteX.
% \newcommand{\packagename}[1]{{\normalfont\sffamily#1}}
% \newcommand\ProLiteX{\textsf{ProLiteX}}
% \newcommand\prolitex{\packagename{prolitex}}
% \newcommand{\ie}{i.\,e.}
%
%^^A The title page
% \title{The \ProLiteX{} package}
% \date{2010/05/07}
% \author{Ralf Hemmecke}
% \maketitle
% \begin{abstract}
%   The \prolitex{} package supports literate programming by providing
%   a new environment |chunk| in which the program code is to be
%   written. It provides means similar to \packagename{noweb} to
%   reference other code chunks.
%
%   Since \prolitex{} inherits all of its power from the
%   \packagename{listings} package by Casten Heinz, all features of
%   \packagename{listings} are available.
% \end{abstract}
%
%^^A
% \section{User's guide}
% There are currently many programs that support literate programming,
% so why here is another one?
%
% For quite some time I was satisfied with the
% \packagename{noweb}\footnote{\url{http://www.cs.tufts.edu/~nr/noweb/}}
% system. However, a \packagename{noweb} file is not a .tex file
% itself. The \packagename{noweb} sytem comes with a program |noweave|
% in order to translate a \packagename{noweb} file into a .tex file.
% Unfortunately, \packagename{noweb} is not installed on every
% computer so that for example systems like
% Axiom\footnote{\url{http://axiom-developer.org/}} and
% FriCAS\footnote{\url{http://fricas.sourceforge.net/}} used to
% include the \packagename{noweb} sources in their distribution and
% build noweb on systems where it was not available. Unfortunately,
% portability is always an issue.
%
% \ProLiteX{} gets rid of the weave part by providing means to write
% literate programming files as pure \LaTeX{}. In addition, the style
% file \packagename{noweb.sty} is no longer referenced, but instead
% simply replaced by the few command contained in the file you are
% currently reading.
% The tangle part is done by a lightweight script (not depending on
% \packagename{noweb}), but that will be described somewhere else.
%
% \ProLiteX{} came into existence shortly after I discovered the
% package \packagename{newfangle}^^A
% \footnote{\url{http://savannah.nongnu.org/projects/newfangle/}} by
% Sam Liddicott. The package \packagename{newfangle} already gets rid
% of the \packagename{weave} step. I actually never used it, but I
% liked the idea that instead of putting the source code into ^^A
% \begin{verbatim}
% <<CHUNKNAME>>=
% program source code
% @\end{verbatim}
%^^A
% as is done in \packagename{noweb}, \packagename{newfangle} builds on
% the very powerful \packagename{listings} package of Carsten Heinz
% and provides a |\Chunk| command, a |\chunkref| and an escape
% mechanism using |=<| and |>|, so that literary programming chunks
% look like in the following example.
%^^A
% \begin{verbatim}
% \Chunk{CHUNKNAME}
% \begin{lstlisting}
% write your code here
% for a in 1..9 repeat {
%   =<\chunkref{CHUNKNAME to be included}>
%   =<\chunkref{other CHUNKNAME}>
% }
% \end{lstlisting}\end{verbatim}
%^^A
% That (except for the |=<| and |>| delimiters) is, of course a more
% natural \LaTeX{} syntax to include code chunks.^^A
% \footnote{Actually, \packagename{doc.dtx} together with
%   \packagename{docstrip.dtx} is a \LaTeX-only solution, but it
%   misses the concept of referencing and including other chunks
%   during the \packagename{weave} (\ie, the \packagename{docstrip})
%   step.}
% The reference delimiters |=<| and |>| appear inside the code and
% thus having them look like \LaTeX{} or any construct of the language
% that appears in the listing is unnatural anyway.
%
% After reading the \packagename{newfangle} sources, I was not
% completely satisfied with it. In particular, the way that
% \packagename{newfangle} references other code chunks by
%^^A
% \begin{verbatim}
% =<\chunkref{other-chunk}>\end{verbatim}
%^^A
% was not exactly the beauty and simplicity I wanted. Furthermore the
% chunkname had to be given by means of the macro |\Chunk| before an
% actual |lstlisting| environment. Unfortunately,
% \packagename{newfangle} also employs \packagename{noweb.sty} to
% typeset the documentation.^^A
% \footnote{The package \packagename{newfangle} also has a quite
% sophisticated tangle step that allows for parametrized chunks and
% thus turning the tangle step into a parametrized macro expansion
% step. \ProLiteX{} tangle does not do that.
% If you want parametrized chunks, use \packagename{newfangle}.
% Oh sorry, that was only a side remark. The |.dtx| file does not deal
% with the tangle part.}
%
% Reading the sources of the \packagename{listings} package (v1.4) is
% a pleasure. They are wonderful. The \prolitex{} package would not
% exist without \packagename{listings}. The |chunk| environment that
% it introduces is, in fact, nothing more than the |lstlisting|
% environment itself, just under another name in order to be able to
% extract code chunks by an appropriate script. Additionally, inside
% the program code one can refer to other code chunks in a way
% identical to \packagename{noweb}, namely using
% \begin{verbatim}
% <<other-chunk>>\end{verbatim}
% inside a |chunk| environment. This feature builds on
% \packagename{listings}' |escapeinside| facility. However, since the
% only command that is needed is |\chunkref|, \prolitex{} provides a
% |chunkref| key that basically works like |escapeinside| and allows
% to choose the delimiters arbitrarily, but also includes a call to
% the |\chunkref| command.
%
% \DescribeEnv{chunk}
% In \prolitex{}, one simply puts the code chunks into
%^^A
% \begin{verbatim}
% \begin{chunk}{CHUNKNAME}
% write your code here
% for a in 1..9 repeat {
%   <<CHUNKNAME to be included>>
%   <<other CHUNKNAME>>
% }
% \end{chunk}\end{verbatim}
%^^A
% where |CHUNKNAME| can be any text that can appear as an argument to a
% \LaTeX{} command, \ie, something like
%^^A
% \begin{verbatim}
% compute $\sum_{k=1}^n \frac{1}{k}$\end{verbatim}
%^^A
% is perfectly valid as |CHUNKNAME|.
% Of course, chunk names longer than one line do not really make sense
% and are, therefore, not supported.
%
% \DescribeEnv{chunkref}
% The delimiters \texttt{<<} and \texttt{>>} can be adjusted by
% providing new delimiters (here \verb'==|' and \verb'|==') as one
% would do for the |escapeinside| key of the listings package.
%^^A
% \begin{verbatim}
% \begin{chunk}[chunkref={==|}{|==}]{CHUNKNAME}\end{verbatim}
%^^A
% One can globally set the chunkref delimiters via
%^^A
% \begin{verbatim}
% \lstset{chunkref={==|}{|==}\end{verbatim}
%^^A
% Being able to change the |chunkref| delimiters according to the
% contents of the code chunk, there is no reason to introduce an
% escape mechanism as is done in \packagename{noweb}.
%
% \DescribeMacro{\chunkref}
% Inside ordinary text one can refer to a chunk by using
%^^A
% \begin{verbatim}
% \chunkref{some-chunk-name}\end{verbatim}
%^^A
% Then the whole name together with its number will be printed.
% There is currently no way to print the number of the chunk alone.

% \DescribeEnv{captionstyle}
% Each program chunk appears in the output by its name preceding
% the actual code listing. It appears as a caption whose style can
% be given using the |captionstyle| key. The argument should be a
% \LaTeX-command that takes one argument (the caption preceded by a
% the chunk number) and does something with it.
% For example, if the \packagename{color} package is loaded
%^^A
% \begin{verbatim}
% \lstset{captionstyle=\color{red}}\end{verbatim}
%^^A
% prints the chunk name in red. The first number is a sequential
% number that is increased by every new |chunk| environment.^^A
% \footnote{Actually, by every |lstlisting| like environment, since
% |chunk| and |lstlisting| use the same counter.}
% The number after the chunk name (inside angle brackets) is the number
% of the first chunk with that name.
%
% A quite useful thing is highlighting the code background by
% ordinary features of the \packagename{listings} package.
%^^A
% \begin{verbatim}
% \lstset{backgroundcolor=\color[rgb]{1,1,0.7}}\end{verbatim}
%^^A
%
% Needless to mention that additionally loading \packagename{hyperref} via
%^^A
% \begin{verbatim}
% \usepackage{hyperref}\end{verbatim}
%^^A
% produces hyperlinked cross references of the code chunks.
%^^A
%
% \section{Implementation}
% \subsection{Version, Date and \ldots}
% The main functionality of the chunk environment heavily builds on
% the listings package of Carsten Heinz.
% prolitex.sty currently works with listings v1.3 and v1.4.
%    \begin{macrocode}
\RequirePackage{listings}[2004/09/07]
%    \end{macrocode}
%
% Let's give \prolitex{} a date and a version and a name. And state
% that we build on \LaTeX{}.
%    \begin{macrocode}
\NeedsTeXFormat{LaTeX2e}
\def\filedate{2010/05/07}
\def\fileversion{0.1}
\AtEndOfPackage{\ProvidesPackage{prolitex}
  [\filedate\space\fileversion\space(Ralf Hemmecke)]}
%    \end{macrocode}
%
% \begin{macro}{\prolitex@other}
% The command |\prolitex@other| just helps to make shorter definitions
% later. We need a way to treat an argument in such a way that it can be
% used for the definition of a command, \ie, inside the |\csname|,
% |\endcsname| pair. The |\@onelevel@sanitize| command defined in
% \LaTeX's \packagename{ltdefns.dtx} does exactly that. It takes a
% control sequence and turns its contents into letters of category
% code 12 (other).
% \begin{verbatim}
% \prolitex@other\somecommand{ARG}\end{verbatim}
% will first turn all characters in ARG into characters with
% category code 12 (other) and then invoke |\somecommand| on this
% translated ARG.
%    \begin{macrocode}
\def\prolitex@other#1#2{\def\prolitex@value{#2}%
  \@onelevel@sanitize\prolitex@value
  \expandafter#1\expandafter{\prolitex@value}}
%    \end{macrocode}
% \end{macro}
%
% \subsection{The chunk environment}
% \begin{environment}{chunk}
% Define the chunk environment in a similar way as the lstlistings
% environment.
%    \begin{macrocode}
\lstnewenvironment{chunk}[2][]{%
  \def\lst@makecaption{\prolitex@makecaption}
  \def\prolitex@value{#2}\@onelevel@sanitize\prolitex@value
  \xdef\lst@intname{\prolitex@value}%
  \lstset{numbers=left,firstnumber=auto,breaklines=false,
    basicstyle=\ttfamily,caption={#2},#1}%
  \csname\@lst @SetFirstNumber\endcsname
}{\csname\@lst @SaveFirstNumber\endcsname}
%    \end{macrocode}
% \end{environment}
%
% \subsection{The chunkref key}
%^^A |chunkref| is not really an environment, but we don't bother do
%^^A define a lstkey environment as is done in the listings package.
% \begin{environment}{chunkref}
% We define a new key |chunkref| that basically behaves like
% \packagename{listings}' own |escapeinside| with the only exception
% that we collect the characters inside the delimiters and consider
% them as an argument to the |\chunkref| command.
%    \begin{macrocode}
\lst@Key{chunkref}{}{\lstKV@TwoArg{#1}%
    {\let\lst@DefEsc\@empty
     \ifx\@empty##1@empty\else \ifx\@empty##2\@empty\else
         \def\lst@DefEsc{\prolitex@EscapeRef{##1}{##2}}%
     \fi\fi}}
%    \end{macrocode}
% Define the default chunk delimiters.
%    \begin{macrocode}
\lstset{chunkref={<<}{>>}}
%    \end{macrocode}
% \begin{macro}{\prolitex@EscapeRef}
% This command is a replacement for |\lst@Escape| from the
% \packagename{listings} package. However, we only have two instead
% of four parameters. We also dropped the calls to
% |\lst@escapebegin| and |\lst@escapeend| in order to avoid any
% overlap between simultaneous |escapeinside| and |chunkref| in the
% same code chunk (which is allowed but probably makes no sense).
%    \begin{macrocode}
\gdef\prolitex@EscapeRef#1#2{%
%    \end{macrocode}
% The macro |\prolitex@EscapeRef| finds the starting delimiter |#1|
% and remember all characters up to |#2| and finally calls |\chunkref|
% on the collected characters. We assume that chunk names fit on one
% line and do not contain a |\par|.
%
% \begin{macro}{\prolitex@process}
% By means of the local macro |\prolitex@process| the ending
% delimiter is found. That is a very simplified version of its
% equivalent part from |\lst@Escape|. Things will completely go
% wrong if the ending delimiter is not found. The |escapeinside| of
% the \packagename{listings} package is actually more relaxed with
% non-existing ending delimiters. However, the goal for \prolitex{} is
% to be used for literate programming and thus no ending delimiter
% means an error for the tangle machinery.
%    \begin{macrocode}
  \gdef\prolitex@process##1#2{\gdef\prolitex@eaten{##1}%
    \expandafter\chunkref\expandafter{\prolitex@eaten}%
    \lst@LeaveAllModes\lst@ReenterModes
    \lst@newlines\z@ \lst@whitespacefalse}%
%    \end{macrocode}
% \end{macro}
% The following code looks for the starting delimiter and then starts
% |\prolitex@process| from above, it is basically a copy of the
% corresponding code from |\lst@Escape| only that we start the above
% |\prolitex@process| in order to find the ending delimiter.
%    \begin{macrocode}
  \lst@CArgX #1\relax\lst@CDefX
    {}%
    {\lst@ifdropinput\else
      \lst@TrackNewLines\lst@OutputLostSpace \lst@XPrintToken
      \lst@InterruptModes
      \lst@EnterMode{\lst@TeXmode}{\lst@modetrue}%
      \expandafter\prolitex@process
      \fi}%
    {}}
%    \end{macrocode}
% \end{macro}
% \end{environment}
%^^A
% \subsection{Reference to a Code Chunk}
% \begin{macro}{\chunkref}
% This is a user definable command that determines the appearance of a
% code chunk reference.
%    \begin{macrocode}
\newcommand{\chunkref}[2][1]{\textrm{%
  \ensuremath{\langle}#2 \prolitex@ref{#2:#1}\ensuremath{\rangle}}}
%    \end{macrocode}
% \end{macro}
% \begin{macro}{\prolitex@ref}
% The macro is essentially calling |\ref| with the argument turned
% into characters of category code 12 (other).
%    \begin{macrocode}
\def\prolitex@ref#1{\prolitex@other\ref{#1}}
%    \end{macrocode}
% \end{macro}

% \subsection{Defining a Code Chunk Label}
% There are two situations where we have to deal with labels, namely
% when they are defined by the |\label| command and when they are read
% in from the |.aux| file.
%^^A
% \subsubsection{Definition in the .tex file}
% \begin{macro}{\prolitex@label}
% The macro is essentially calling |\label| with the argument turned
% into characters of category code 12 (other).
%
% The original |\label| command should work well if all chunk names
% were different. However, they are usually not. A |chunk| environment
% continues a previous environment with the same name.
% In order to cope with that, we introduce a label generator, \ie, the
% first chunk with name |NAME| gets the label |NAME:1|. The following
% chunks with the same |NAME| get |NAME:2|, |NAME:3|, etc. as their
% respective label names.
%    \begin{macrocode}
\def\prolitex@label#1{\prolitex@other\prolitex@other@label{#1}}
%    \end{macrocode}
% \end{macro}
%
% \begin{macro}{\prolitex@other@label}
% The following macro introduces one command |sc@NAME| for each
% chunk name |NAME|. Its value is the number that is described above.
% We use two \TeX{} tricks here.
% \begin{enumerate}
% \item |\csname sc@X\endcsname| expands to |\relax| if |\sc@X| is an
%   undefined control sequence. So we don't need |\@ifundefined|.
% \item When \TeX{} reads a number, it continues until it encounters a
%   non-digit.
% \end{enumerate}
% Since |\csname sc@#1\endcsname| will be either undefined or expand
% to a number, by the first line, we initialize the counter
% |\@tempcnta| to either 0 or the last already assigned number.
%    \begin{macrocode}
\def\prolitex@other@label#1{%
  \@tempcnta 0\csname sc@#1\endcsname\relax
  \advance\@tempcnta\@ne
  \expandafter\xdef\csname sc@#1\endcsname{\the\@tempcnta}
  \def\@tempa{#1:}%
  \expandafter\expandafter\expandafter\label
  \expandafter\expandafter\expandafter{%
    \expandafter\@tempa\the\@tempcnta}}
%    \end{macrocode}
% \end{macro}
%^^A
% \subsubsection{Reading the label from the .aux file}
% \begin{macro}{\newlabel}
% We just make sure that nothing bad can happen with our labels.
% Since we allow a broader character set for our labels, we must make
% sure that the category codes are set to 12 (other) when they are
% read by |\newlabel|.
% There are two places where the |.aux| file will be read.
% \begin{enumerate}
% \item Before |\begin{document}| a possibly already existing |.aux|
%   file is read.
% \item At the time when |\end{documnent}| is executed the command
%   |\@newl@bel| is redefined by \LaTeX{} to mean
%   |\@testdef|\footnote{|@\@testdef| basically checks whether all
%   labels have proper definitions.} and the newly generated |.aux|
%   file is read.
% \end{enumerate}
% Redefining |\newlabel| directly in \packagename{prolitex.sty} only
% works for the first inclusion of the |.aux| file. For the second
% inclusion we would have to redefine |\@testdef|. Since the latter
% is too deep inside the heart of \LaTeX{}, we decided to write the
% redefinition of |\newlabel| into the |.aux| file itself.
%    \begin{macrocode}
\AtBeginDocument{%
  \protected@write\@auxout{}{%
    \string\let\string\prolitex@newlabel@save\string\newlabel^^J%
    \string\def\string\newlabel\string#1{%
      \string\def\string\prolitex@value{\string#1}%
      \string\@onelevel@sanitize\string\prolitex@value^^J%
      \string\expandafter\string\prolitex@newlabel@save\string\expandafter{%
        \string\prolitex@value}}}}
%    \end{macrocode}
% \end{macro}
%^^A
%
% \subsection{Captions}
%^^A |captionsstyle| is not really an environment, but we don't bother do
%^^A define a lstkey environment as is done in the listings package.
% \begin{environment}{captionstyle}
% We define a new key |captionstyle| that allows to determine the
% appearance of the caption of a chunk environment.
%    \begin{macrocode}
\lst@Key{captionstyle}{}{\def\prolitex@captionstyle{#1}}
%    \end{macrocode}
% \end{environment}
% Define the defaultstyle for the chunk caption.
%    \begin{macrocode}
\lstset{captionstyle=\textrm}
%    \end{macrocode}
%
% \begin{macro}{\chunkcaptionadjust}
% We want the chunk caption left adjust, so we don't have the |\hfil|
% in front of |#1|.
%    \begin{macrocode}
\def\prolitex@captionadjust#1{#1\hfil}
%    \end{macrocode}
% \end{macro}
%
% \begin{macro}{\prolitex@makecaption}
% Adjust the appearance of the caption (left adjusted) and ignore the
% first and second parameters since we know they must be
% |\fnum@lstlisting| and |\lst@caption|.
%
% In order to keep the |lstlisting| environment and the |chunk|
% environment separate, we do not redefine |\fnum@lstlisting| but
% rather use |\thelstlisting| directly.
%
% We assume that chunk names are less than one line in size.
%
% We use the command |\prolitex@makecaption| locally as
% |\@makecaption| in the definition of the |chunk| environment.
%
% Our own definition of the |\@makecaption| command is necessary,
% because our first attempt that worked for \packagename{listings} 1.3
% broke with version 1.4, since in \packagename{listings}' 1.3 definition
% of |\lst@MakeCaption| appears the line
% \begin{verbatim}
% \lst@makecaption\fnum@lstlisting\lst@caption\end{verbatim}
% whereas in version 1.4 an |\ignorespaces| has been added.
% \begin{verbatim}
% \lst@makecaption\fnum@lstlisting{\ignorespaces \lst@caption}\end{verbatim}
% That interferes badly with our need to expand |\lst@caption| once
% and use the characters for printing and for determining the chunk
% number. Thus, here is a new one.
%    \begin{macrocode}
\long\def\prolitex@makecaption#1#2{%
  \vskip\abovecaptionskip
  \expandafter\prolitex@label\expandafter{\lst@caption}
  \sbox\@tempboxa{\prolitex@captionstyle{\thelstlisting
      \expandafter\prolitex@chunkrefnav\expandafter{\lst@caption}}}%
  \global \@minipagefalse
  \hb@xt@\hsize{\prolitex@captionadjust{\box\@tempboxa}}%
  \vskip\belowcaptionskip}
%    \end{macrocode}
% \end{macro}
%
% \begin{macro}{\prolitex@chunkrefnav}
% After execution of |\prolitex@label|, the counter |\@tempcnta| has
% an appropriate value. We only have to test whether the previous and
% next labels have been assigned and output the appropriate navigation
% code.
%    \begin{macrocode}
\def\prolitex@chunkrefnav#1{%
  \chunkref{#1}
  \def\prolitex@value{#1}\@onelevel@sanitize\prolitex@value
  \advance\@tempcnta by -1
  \@ifundefined{r@\prolitex@value:\the\@tempcnta}{}{%
    \ensuremath{\triangleleft}\ref{\prolitex@value:\the\@tempcnta} }
  \advance\@tempcnta by 2
  \@ifundefined{r@\prolitex@value:\the\@tempcnta}{}{%
    \ref{\prolitex@value:\the\@tempcnta}\ensuremath{\triangleright}}}
%    \end{macrocode}
% \end{macro}
